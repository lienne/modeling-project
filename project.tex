\documentclass{article}
\usepackage{graphicx}
\usepackage[margin=1in]{geometry}
\usepackage{amsmath}
\title{Mathematical Modeling Project}
\author{Tyler Lukasiewicz, Liana Severo, Caitlin Buttery}
\begin{document}
\maketitle
\abstract{
babkabkaka mea quidam pericula appellantur, ne quo vitae recteque. Duo ullum deserunt definitiones ea, ex has vocent constituam inciderint, at novum philosophia has. Summo facete audire eu pro, mazim tritani tibique ne quo. Justo deterruisset reprehendunt qui cu, nam te debet oblique incorrupte. Usu nominavi copiosae patrioque et, nec ferri constituto ei.

}
\section{Introduction}
\label{sec:Introduction}
Lorem ipsum dolor sit amet, id mea quidam pericula appellantur, ne quo vitae recteque. Duo ullum deserunt definitiones ea, ex has vocent constituam inciderint, at novum philosophia has. Summo facete audire eu pro, mazim tritani tibique ne quo. Justo deterruisset reprehendunt qui cu, nam te debet oblique incorrupte. Usu nominavi copiosae patrioque et, nec ferri constituto ei.


\section{Research and Methods}
\label{sec:Main part}

\subsection{Immune System Response to a Single Viral Strain}
\begin{equation}
    \begin{split}
        \dot v &= v(r-ax) = f_1(v,x), \\
        \dot x &= -bx + cv = f_2(v,x)
    \end{split}
\end{equation}

Here, $v$ is the viral strain. $x$ is the specific immune system response to the strain. $r$ is the rate at which the virus reproduces. $a$ is the rate at which the immune cells destroy the virus. $b$ is the rate at which the immune cells die off. $c$ is rate at which the immune cells reproduce, which is dependent on the number of viruses present, $v$.  
\subsubsection{Getting our eigenvalues}
First we take the Jacobian matrix of our system, which is
\begin{equation}
    J(v,x) =
    \begin{pmatrix}
        r-ax    & -av \\
        c       & -b
    \end{pmatrix}
\end{equation}
Then we find the characteristic equations by evaluating the Jacobian at the two fixed points of our system $(0,0)$ and $(\alpha,\beta)$, where $\alpha = \frac{br}{ac} $ and let $\beta = \frac{r}{a} $
\begin{equation}
    \begin{split}
        &J(0,0) - \lambda I = \lambda^2 + \lambda(b-r) -br = 0\\
        &\implies \lambda_{1,2} = r,-b\\
        &J(\alpha,\beta)  - \lambda I =  \lambda^2 + \lambda \gamma + \delta\\ 
        &\implies \lambda_{1,2} = \frac{-\gamma \pm \sqrt{\gamma^2 - 4\delta}}{2} \\
        \text{where} \\
        &\gamma =  b - r + a\beta \text{, and } \delta = ac\alpha + ab\beta -rb
    \end{split}
\end{equation}
\subsubsection{Modeling the System With One Viral Strain}
We will now model our system of equations with conditions $r = 2.4, a = 2, b = 0.1, \text{ and }c = 1.$. We will also assume that we are starting with no viruses and no immune response.
\label{sub:modeling the sytem}

The fixed point $(0,0)$ coresponds to the eigenvalues $\lambda_1 = 2.4, \lambda_2 = -.1$, which implies that $(0,0)$ is a saddle point. The fixed point $(\alpha,\beta) = (.12,1.2)$ results in eigenvalues $\lambda_1 = -.05 + 4.873i , \lambda_2 = -.05 - 4873i$, which implies that the point $(.12,1.2)$ is a spiral sink. So in this system both the viral strain and immune response will begin oscillating dramatically and then as time approaches infinity, they will settle to stable values. This is illustrated in figure \ref{fig:hiv1}
\label{sub:Determining Stability of Fixed points}
\begin{figure}[!tbp]
    \centering
    \begin{minipage}[b]{0.4\textwidth}
	    \caption{\textbf{\textit{Population sizes for a single virus strain with r = 2.4, a = 2, b = 0.1, c = 1.}}}
	    \includegraphics[scale=.25]{imgs/hiv_graph1.png}
	    \label{fig:hiv1}
	\end{minipage}
	\hfill
	\begin{minipage}[b]{0.4\textwidth}
		\caption{\textbf{\textit{Population sizes for multiple strains  with r = 2.4, a = 2, b=0.1,c=1,q=2.4,k=1.}}}
		\includegraphics[scale=.25]{imgs/hiv_graph2.png}
		\label{fig:hiv2}
	\end{minipage}
\end{figure}

\subsection{Modeling the System with Multiple Viral Strains}
Suppose there are $N$ strains of the virus.  The $i$th strain of the virus $v_i$ and the immunal reaction $x_i$ to it can be modeled by the system of equations
\begin{equation}
    \begin{split}
        \dot v_i &= v_i(r - ax_i), \\
        \dot x_i &= -bx_i + cv_i
    \end{split}
\end{equation}
This adds a degree of randomness to our behavior, as new viruses can appear at any point in time. We may begin with only one virus, which can proceed to mutate as it pleases, or we may begin with many strains. Each new viral strain should result in a new immune response. Eventually, a global immune response will take care of all $N$ viral strains regardless of mutation or rate of mutation. This global response can be modeled by the system of equations
\begin{equation}
    \begin{split}
        \dot v_i &= v_i(r - ax_i - qz) \\
        \dot x_i &= -bx_i + cv_i \\
        \dot z &= kv - bz
    \end{split}
\end{equation}
Where $z$ is the cross reactive response that decays at rate $b$.  $v = \sum^{N}_{i=1} v_i$ is the total viral load. $q$ is the rate at which the virus evades the global response, and $k$ is the rate at which the global response grows in comparison to the number of preheat viral strains.


\subsubsection{Getting Eigenvalues}
In this case we are going to be introducing a the variable:
\[
V = \sum_{i = 1}^N v_i
\]
We will assume that $V$ is constant at each iteration. Thus our new system of equations becomes 
\begin{equation}
	\begin{split}
		\dot v &= v(r - ax - qz) \\
		\dot x &= -bx + cv \\
		\dot z &= kV - bz
	\end{split}
\end{equation}
First we take the Jacobian of the system of equations. 
%the jacobian
\begin{equation}
    J=
    \begin{pmatrix}
        -ax-qz+r    &-va    &-qv    \\
        c           &-b     &0      \\
        0           &0       &-b
    \end{pmatrix}
\end{equation}
%the characteristic equation
Next we find the characteristic equation
\begin{equation}
    \begin{split}
        | J - \lambda I | &= \\ 
        &-a{b}^{2}x-abcv-2\,ab\lambda\,x-ac\lambda\,v-a{\lambda}^{2}x-{b}^{2}qz \\
        &-2\,b\lambda\,qz-{\lambda}^{2}qz-{b}^{2}\lambda+{b}^{2}r-2\,b{\lambda}^{2}+2\,b\lambda\,r-{\lambda}^{3}+{\lambda}^{2}r
    \end{split}
\end{equation}

solving for lambda, we get the eigenvalues

\begin{equation}
    \begin{split}
        \lambda_1 &=  -b\\
        \lambda_{2,3} &= -1/2\,ax-1/2\,qz-b/2+r/2 \\
        &\pm 1/2\,\sqrt {{a}^{2}{x}^{2}+2\,axqz+{q}^{2}{z}^{2}-2\,abx-4\,acv-2\,axr-2\,bqz-2\,qzr+{b}^{2}+2\,br+{r}^{2}}         
    \end{split}
\end{equation}


\subsubsection{Modeling The Equation}

We will be studying the behavior of this equation with the values

\begin{equation}
    \begin{split}
        r = 2.4, a = 2, b=0.1,c=1,q=2.4,k=1
    \end{split}
    \label{eq:vals}
\end{equation}

Our first step is to determine the fixed points of the system of equations. In this case there are two.

%fixed points of equation
\begin{equation}
	\begin{split}
		&\text{Fixed point 1: } \quad v=0,x=0,z={\frac {kV}{b}} \\
		&\text{Fixed point 2: } \quad v=-{\frac {qkV-br}{ac}},x=-{\frac {qkV-br}{ab}},z={\frac {kV}{b}} 
	\end{split}
	\label{eq:fixed}
\end{equation}

Our observations show that our first fixed point will act as a source and the second will act as a spiral sink. This is demonstrated in figure \ref{fig:hiv2}





\subsection{HIV Section}

\subsection{Ebola Section}

\section{Conclusion}
\label{sub:Conclusion}
Lorem ipsum dolor sit amet, id mea quidam pericula appellantur, ne quo vitae recteque. Duo ullum deserunt definitiones ea, ex has vocent constituam inciderint, at novum philosophia has. Summo facete audire eu pro, mazim tritani tibique ne quo. Justo deterruisset reprehendunt qui cu, nam te debet oblique incorrupte. Usu nominavi copiosae patrioque et, nec ferri constituto ei.



\begin{thebibliography}{9}
\bibitem{latexcompanion} 
Michel Goossens, Frank Mittelbach, and Alexander Samarin. 
\textit{The \LaTeX\ Companion}. 
Addison-Wesley, Reading, Massachusetts, 1993.
\end{thebibliography}


\end{document}
